% 2024
\geometry{left=1.9cm,right=1.9cm,top=1.76cm,bottom=1.8cm}
\pagestyle{empty}

 %不让那些section和subsection自带标号,标号格式自己掌握
 \setcounter{secnumdepth}{-2}

 %Ms Word 的蓝色和latex xcolor包预定义的蓝色不一样。通过屏幕取色得到。
 \definecolor{MsBlue}{RGB}{0,112,192} 
 
 % Renaming floats with babel
 \addto\captionsenglish{
     \renewcommand{\contentsname}{目录}
     \renewcommand{\listfigurename}{插图目录}
     \renewcommand{\listtablename}{表格}
     %\renewcommand{\refname}{\sihao 参考文献}
     %这几个字默认字号稍大,改成四号字,楷书,居左(默认居中) 根据喜好自行修改,官方模板未作要求
     \renewcommand{\refname}{\sihao \kaishu \leftline{参考文献}} 
     \renewcommand{\abstractname}{摘要}
     \renewcommand{\indexname}{索引}
     \renewcommand{\tablename}{表}
     \renewcommand{\figurename}{图}
     } %把Figure改成‘图’,reference改成‘参考文献’。如此处理是为了避免和babel包冲突。
 %定义字号
 
 \newcommand{\chuhao}{\fontsize{42pt}{\baselineskip}\selectfont}
 \newcommand{\xiaochuhao}{\fontsize{36pt}{\baselineskip}\selectfont}
 \newcommand{\yihao}{\fontsize{26pt}{\baselineskip}\selectfont}
 \newcommand{\erhao}{\fontsize{22pt}{\baselineskip}\selectfont}
 \newcommand{\xiaoerhao}{\fontsize{18pt}{\baselineskip}\selectfont}
 \newcommand{\sanhao}{\fontsize{16pt}{\baselineskip}\selectfont}
 \newcommand{\sihao}{\fontsize{14pt}{\baselineskip}\selectfont}
 \newcommand{\xiaosihao}{\fontsize{12pt}{\baselineskip}\selectfont}
 \newcommand{\wuhao}{\fontsize{10.5pt}{\baselineskip}\selectfont}
 \newcommand{\xiaowuhao}{\fontsize{9pt}{\baselineskip}\selectfont}
 \newcommand{\liuhao}{\fontsize{7.875pt}{\baselineskip}\selectfont}
 \newcommand{\qihao}{\fontsize{5.25pt}{\baselineskip}\selectfont}
 %字号对照表
 %二号 21pt
 %四号 14
 %小四 12
 %五号 10.5
 
 %设置行距 1.5倍
 \renewcommand{\baselinestretch}{1.5}
 \XeTeXlinebreaklocale "zh"           % 中文断行

 %\setmainfont{Times}
 \setCJKmainfont{KaiTi}
