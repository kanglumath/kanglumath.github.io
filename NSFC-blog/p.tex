% 国家自然科学基金NSFC海外优青申请书正文模板(2024版)

% 声明:
% 注意!!!非国家自然科学基金委官方模版!!!由个人根据官方MsWord模版制作。本模版的
% 作者尽力使本模版和官方模版生成的PDF文件视觉效果大致一样,然而,并不保证本模版有用,
% 也不对使用本模版造成的任何直接或间接后果负责。 强烈建议自己对照官方MsWord模板确认格式
% 和文字是否一致,尤其是蓝字部分。

%本模版可以自由修改以满足用户自己的需要。但是如果要传播本模版,则只能传播未经修改的版本。
%不得将本模版用于商用或获取经济利益。
%使用本模版意味着同意上述声明。


%默认小四号字。允许楷体粗体。
\documentclass[12pt,UTF8,AutoFakeBold=4,a4paper]{ctexart} 
\usepackage{amsmath,amssymb,graphicx,mathrsfs,bm}
\include{pkgs}
\input{cmds}


%%%% 正文开始 %%%%
\begin{document}

\begin{center}
{\sanhao \CJKfontspec{SimHei} 报告正文 
\bfseries \fontspec{Times New Roman} Main Body of Proposal}
\end{center}

{\sihao \kaishu  
参照以下提纲撰写,要求内容翔实、清晰,层次分明,标题突出。}

\bigskip

{\sihao \fontspec{Times New Roman} The proposal shall be written in accordance with the following outline, with informative content, clear structure, and prominent titles.}

\bigskip

{\sihao \kaishu \color{MsBlue} \bfseries 请勿删除或改动下述提纲标题及括号中的文字。}

\bigskip

{\sihao \color{MsBlue} \fontspec{Times New Roman} Please do not delete or change the title of the outline and the words in brackets.}

% \vskip -5mm


{\color{MsBlue} \subsection{\sihao \kaishu \qquad \ 
\textbf{(一)主要学术成绩}{\normalfont(建议不超过4000字)}}}

%\medskip

{\color{MsBlue} \xiaosihao \fontspec{Times New Roman} 
\textbf{Major academic achievements(no more than 4000 words)}}

\medskip

{\sihao \kaishu \color{MsBlue} 着重阐述所取得研究成果的创新性、科学价值及本人贡献等。}

\medskip

{\color{MsBlue} \xiaosihao \fontspec{Times New Roman} In this part, you shall focus on the innovativeness and scientific value of the research results, and your personal contribution.}

% \newpage
% \setlength{\bibsep}{0.0pt}
% \bibliographystyle{gbt7714-numerical}
% \bibliography{p}
% \newpage

%可以通过类似的命令微调行距以使得排版美观
% \vskip -5mm 

申请人的研究方向是量子群表示论和量子可积系统,主要集中在研究杨代数和仿射量子群及它们的余理想子代数---扭杨代数和仿射$\imath$量子群---在几何与量子可积系统系统中的应用。这是表示论的最新领域和核心方向之一,与典型李代数的表示、Slodowy切片、有限W代数、Coulomb分支等联系紧密。\textbf{申请人的标志性成果有}:

\textbf{(1)通过对仿射$\imath$量子群做退化自1990年以来首次引入了所有分裂型扭杨代数的Drinfeld表示,并给出来其PBW基;构造了仿射$\imath$量子群的R-matrix实现和Drinfeld实现之间的具体同构,并以此给出了仿射$\imath$量子群在Drinfeld实现的PBW基。}

\textbf{(2)系统地研究了超杨代数,包括引入其$q$-特征和斜表示,构造Drinfeld函子来建立其Schur-Weyl对偶性,并通过计算斜表示的$q$特征证明了它们满足量子Jacobi-Trudi等式,从而证明了数学物理里被广泛使用的CBR公式。}

\textbf{(3)引入了ABCD型李超代数对称的量子可积系统的Bethe方程的reproduction procedure及其对应的QQ系统,证明了在A型时reproduction procedure的不变有理差分算子包含了传递矩阵的特征值,完整刻画了超$\mathrm{Y}(\mathfrak{gl}_{1|1})$ XXX自旋链的Bethe代数及证明其完美可积性。}

\textbf{申请人在Comm. Math. Phys.,JHEP,IMRN等国际一流期刊发表和接收了18篇文章,相关工作被美国艺术与科学院院士、MIT教授Etingof,美国艺术与科学院院士、Herman Weyl奖得主、UC Berkely教授E.~Frenkel,Shaw奖得主、ICM一小时报告人Kazhdan教授,量子群创始人之一Jimbo教授,澳大利亚科学院院士、悉尼大学教授Molev,Whitehead奖得主、ICM 45钟报告人约克大学教授Nazarov等多位国际顶尖数学家引用,多次担任Comm. Math. Phys.,Compos. Math.,IMRN,J. Algebra,SciPost Phy.等多个国际期刊审稿人。}

\medskip

现在将代表性工作简介如下。

\textbf{\sihao 1、扭杨代数与仿射$\imath$量子群}

\textbf{A. 扭杨代数的Drinfeld实现}

杨代数(Yangians)最早出现在数学物理中,是在 80 年代国际著名数学物理学家 Faddeev 及其学派关于量子反散射方法的研究中。更一般的杨代数是由菲尔茨奖、沃尔夫奖得主Drinfeld在1985年引进用来给Yang-Baxter方程提供有理函数解的、非常重要的一类量子群。它们在几何、数学物理、表示论有着非常广泛的应用。扭杨代数(twisted Yangians)是杨代数的一类重要的余理想子代数。不同于杨代数是用Dynkin图来分类的,扭杨代数是用对称对或者Satake图来分类的。它们(AI和AII型)最先是由Olshanski在1990年通过R-matrix和Cherednik的反射方程来引进的。另外,扭杨代数有到BCD型李代数的赋值同态,因此\textbf{扭杨代数比BCD型杨代数与BCD型李代数的表示论的联系更加紧密}。

杨代数和仿射量子群的Drinfeld(新)实现是由Drinfeld在1988年引进的。Drinfeld实现对于引进$q$-特征,研究仿射量子群的表示论,以及引进shifted杨代数至关重要。因为李代数可以看作是对角型的对称对,因此可以认为对称对是李代数的推广,进一步扭杨代数是杨代数的推广而$\imath$量子群是量子群的推广。因此,一个非常自然且重要的问题就是扭杨代数是不是也有Drinfeld实现。\textbf{这是一个长期悬而未决的公开问题},甚至在卢明教授和王伟强教授给出仿射$\imath$量子群的Drinfeld实现之前,人们可能认为这样的Drinfeld实现并不存在。

%申请人与王伟强教授,张伟楠博士(arXiv:2308.12254)通过R-matrix实现做高斯分解的方法\textbf{得到了分裂型A型(即AI型)twisted Yangians的Drinfeld实现}。在后续的工作中(arXiv:2406.05067,arXiv:2408.06981),申请者与合作者对仿射$\imath$量子群的Drinfeld实现做退化,从而\textbf{得到了所有分裂型和拟分裂ADE型twisted Yangians的Drinfeld实现}。我们的创新点在于这两种方法独立却又互补。一方面高斯分解帮助我们确定如何在仿射$\imath$量子群那边选取合适的参数做退化。另一方面通过退化仿射$\imath$量子群的Serre关系式,我们得到twisted Yangians的Serre关系式,而这个Serre关系式单独从高斯分解是很难得出一个简单而又优美的公式。

申请人与王伟强教授,张伟楠博士对仿射$\imath$量子群的Drinfeld实现做退化,从而\textbf{得到了所有分裂型和拟分裂ADE型扭杨代数的Drinfeld实现}。我们的创新点在于做退化的方法和我们在预印本的另一种高斯分解法独立却又互补。一方面高斯分解帮助我们确定如何在仿射$\imath$量子群那边选取合适的参数做退化。另一方面通过退化仿射$\imath$量子群的Serre关系式,我们得到扭杨代数的Serre关系式,而这个Serre关系式单独从高斯分解是很难得出一个简单而又优美的公式。另外,我们通过对生成函数形式的关系式做退化而不是生成元,从而得出了更为复杂的BCF型Serre关系式。在证明仿射$\imath$量子群的退化同构于扭杨代数的过程中,我们还证明了扭杨代数的PBW定理及构造其对应PBW基。这进一步说明了扭杨代数是扭流李代数的泛包络代数的平坦形变,保证了我们引进的扭杨代数是合理的。这一工作发表在《Comm. Math. Phys.》。

扭杨代数的Drinfeld实现将提供一系列的应用,比如研究扭杨代数表示论的$q$-特征理论,引进shifted扭杨代数并研究其与几何中Slodowy切片、有限W代数、仿射Grassmannian切片的联系。最近,Topley教授和其学生Tappeiner(arXiv:2406.05492) 利用我们的扭杨代数的 Drinfeld实现研究典型李代数的Slodowy切片,从而极大地推广了Topley教授在2023年发表在《Invent. Math.》的结果。

%,得到了两位审稿人的高度评价:``\textbf{addresses an important problem} in the representation theory of affine
%quantum symmetric pairs that will help \textbf{bring the theories of twisted Yangians and affine ıquantum groups to a level playing field}",``\textbf{was an important open problem for many years} which has been \textbf{resolved in the present paper in many cases}"。

\textbf{B. 仿射$\imath$量子群的R-matrix和Drinfeld实现} 

仿射量子群通常有三种常用的实现方式,即Drinfeld-Jimbo、Drinfeld、和R-matrix实现。不同的实现方式有不同的优势,有些实现方式在研究特定问题如可积系统的对称性、表示论、或与几何之间的联系比其他实现方式更方便。比如说Drinfeld实现是研究仿射量子群的表示论主要工具,它能用来定义$q$-特征,统一地给出有限维不可约表示的分类,定义截断shifted仿射量子群(包括杨代数)并研究其与仿射Grassmannian切片(或更进一步与K-理论版本的Coulomb分支)的联系等。另一方面,Drinfeld-Jimbo和R-matrix实现相比Drinfeld实现能更好地描述仿射量子群的Hopf代数结构,特别是它们的余积(coproduct)。因此\textbf{找到仿射量子群不同实现方式的之间的具体同构与联系是非常重要且有用的工作}。这一系列工作由Beck,Ding-Frenkel,Damiani,Jing-Liu-Molev等教授完成。在此方面最近的推广还包括景乃桓教授、胡乃红教授与合作者关于双参数仿射量子群之间的同构,和张红莲教授与合作者关于B型量子仿射超代数之间的同构等。

扭$q$-杨代数(twisted $q$-Yangians)是Molev-Ragoucy-Sorba通过类似Olshanski的方法用R-matrix实现构造的A型仿射量子群的余理想子代数。它们对应的对称对(或Satake图)是AI和AII型,最近景乃桓教授和张健教授研究了它们的Sklyanin余子式之间的一系列重要的等式。更一般的仿射$\imath$量子群,作为仿射量子群的余理想子代数,是由Kolb教授[Adv. Math. 2014]通过Drinfeld-Jimbo实现推广Letzter教授对于有限型量子对称对而引进的。并且Kolb教授证明了AI和AII型仿射$\imath$量子群分别同构于对应的扭$q$-杨代数并给出Drinfeld-Jimbo实现和R-matrix实现的生成元之间的具体对应。最近,王伟强教授和卢明教授[Adv. Math. 2021]通过仿射$\imath$量子群的辫子群作用构造了一系列新的生成元从而得到了分裂ADE型仿射$\imath$量子群的Drinfeld实现,推广前面Beck和Damiani的工作。这一工作被张伟南博士推广到了分裂BCFG型。但是,仿射$\imath$量子群的R-matrix实现和Drinfeld实现之间的具体同构仍是缺失的。

申请人通过结合Ding-Frenkel早期关于A型仿射量子群的工作和最近与合作者关于扭杨代数的工作,\textbf{给出了仿射$\imath$量子群(AI型)的R-matrix实现和Drinfeld实现的具体同构并给出了生成元之间的对应}。特别地,利用R-matrix实现的PBW定理及该具体同构,申请人\textbf{构造了Drinfeld实现的PBW基及证明了PBW定理}。这一工作将为未来构造其他类型仿射$\imath$量子群的R-matrix和Drinfeld实现的同构提供指导意见。该结果发表在《Int. Math. Res. Not.》。
%,并得到了审稿人的好评“\textbf{presents crucial and important results}”“I \textbf{strongly recommend} it for publication in IMRN”。

\medskip

\textbf{\sihao 2、超杨代数的表示论及Jacobi-Trudi等式}

超对称是现代理论物理非常重要的一个理论框架,而超代数是描述超对称的数学语言。这里超代数是一个$\mathbb Z_2$-分次代数,可以分解为奇、偶两部分。超杨代数是由ICM\,45分钟报告人Nazarov教授引入的,关于杨代数的自然推广,它们在表示论和数学物理里非常重要,有着非常广泛的应用。例如,国立中山大学的彭勇宁教授用超杨代数及其子代数Shifted超杨代数来刻画有限W超代数的显式实现;Nazarov教授利用它的量子超行列式(Berezinian)来刻画线性李超代数的Capelli等式;常浩教授和胡红梅教授深入地研究了超杨代数的量子超行列式;超杨代数也是描述超XXX自旋链的对称性的主要数学工具。因此,\textbf{研究超杨代数的代数结构及其表示论是数学物理和表示论的核心问题}。

申请人与导师Mukhin教授系统地研究了A型超杨代数的表示理论。我们的结果包括引入超杨代数的$q$-特征和斜表示(取决于斜Young图)及计算它们的$q$-特征,从而进一步\textbf{证明了斜表示的$q$-特征满足Jacobi-Trudi等式}。通过对泛R-matrix特定到此Jacobi-Trudi等式,我们证明了对应斜Young图的传递矩阵(transfer matrices,数学物理里非常重要的一类哈密顿算子)也满足Jacobi-Trudi等式,即\textbf{给出了超对称情形的Cherednik-Bazhanov-Reshetikhin (CBR)公式在代数层面的证明}。CBR公式在研究超XXX自旋链的传递矩阵的谱问题中被物理学家广泛应用,我们的结果\textbf{首次给出了其严格的数学证明并提供了组合与表示论背景}。Molev教授和Ragoucy教授证明了量子超行列式是传递矩阵的生成函数。通过使用Jacobi-Trudi等式,我们将这个量子超行列式写成了一个$\mathcal D_1\mathcal D_2^{-1}$的形式,其中$\mathcal D_1$和$\mathcal D_2$分别是类似量子行列式的微分算子。这在超对称情形是一个新的现象。

同时,我们\textbf{构造了超杨代数的Drinfeld函子,从而建立了超杨代数的Schur-Weyl对偶}。Schur-Weyl对偶表明(超)杨代数的模范畴和退化仿射Hecke代数的模范畴是等价的。此外,Drinfeld函子是正则的,且它将退化仿射Hecke代数的单模映到(超)杨代数的零模或单模。这样,我们\textbf{将一些在杨代数里的关于不可约性和Grothendieck环里的等式等结果系统地推广到超杨代数情形}。特别地,我们得到了超杨代数的广义T--系统(差分Hirota方程),这是另一个在数学物理里非常有用的等式。该结果发表在《Int. Math. Res. Not.》。
%并得到了审稿人的好评“The main result (Theorem 5.12) is certainly \textbf{interesting and beautiful}” “it is also \textbf{novel and not an obvious generalization} of a known result for $\mathrm{Y}(\mathfrak{gl}_m)$”。


\medskip

\textbf{\sihao 3、量子可积系统与Bethe ansatz}

\textbf{A. Bethe ansatz和QQ关系式}

在研究量子可积系统中,一个核心的问题是找到一系列可交换的传递矩阵(或Bethe代数,即这些传递矩阵的系数生成的量子群的交换子代数)作用在一个量子群的(通常为有限维不可约)表示的所有特征值与对应的特征向量。传递矩阵可以通过从泛R-matrix特定到具体的表示再取迹得到。这个将数学物理的自旋链推广到了研究量子群的表示论的方法叫做量子反散射方法。最简单的传递矩阵对应自然表示,我们称之为标准传递矩阵,而将更复杂的称为高阶传递矩阵。在理论物理中,有一个非常系统方法来解决传递矩阵的谱问题,叫做Bethe ansatz。Bethe ansatz是由诺贝尔奖得主Bethe在1931年提出来解决Heisenberg链的谱问题的一种方法,而用这一方法来解决Heisenberg链的严格数学证明是由诺尔奖得主杨振宁与合作者在1966年完成。Bethe ansatz的方法是假设一个特殊的关于一些变量(Bethe变量)的有理函数向量,然后计算发现当这些变量满足Bethe方程时,此向量正好是标准传递矩阵的特征向量,称之为Bethe向量。因此,如果我们能够找到Bethe方程的解,那么我们就能得到一个特征向量。Bethe方程是一系列的有分母的代数方程,因此求解这些方程并不平凡,另外求解Bethe方程可能遗漏特殊的奇异解。

申请人和导师Mukhin教授对超XXX自旋链及其Bethe解进了系统地研究。不同于直接求解Bethe方程,我们引入一些新的关于$u$的多项式,这些多项式的根正好是某些Bethe变量按对应李代数的根系来分配。我们发现,在非退化情形,求解Bethe方程等价于求解一系列关于这些多项式的Wronskian方程。在非超对称情形,这些Wronskian方程通常被称为QQ系统,其中的Q对应Baxter的Q算子,满足著名的Baxter的TQ关系。当Bethe方程满足时,这些多项式代表Q算子在对应的Bethe向量的特征值。通过研究QQ系统,我们引入了Bethe方程的reproduction procedure(数学物理中的B\"{a}cklund 变换)。这个reproduction procedure与Weyl群及丛代数的突变密切相关,并且能\textbf{将一个Bethe解变成另一个Bethe解}。给定一个Bethe解,我们引进了一个有理差分算子和population。这里population是所有从这个给定的Bethe解通过不断重复reproduction procedure得到的Bethe解的集合。我们\textbf{证明了population作为一个簇同构于超旗簇,并且同一个population里的不同Bethe解对应同一个有理差分算子的不同的完全线性分解}。这一结果将Bethe向量与超旗簇的点对应起来,因此可以认为是\textbf{超XXX自旋链的一种几何Langlands对应}。我们的结果\textbf{给出了QQ系统和B\"{a}cklund 变换的严格的数学定义}。这一结果发表在《J. Phys. A: Math. Theor.》。我们进一步将类似结果推广到了正交辛李超代数(包括BCD型)的Gaudin模型,对应的有理差分算子变成了伪微分算子。特别地,我们的结果给出了D型非超对称的结果。这一工作发表在《Ann. Henri Poincaré 》。

我们猜想上述引进的有理差分算子在展开成差分算子的级数之后,其系数是(包括高阶)传递矩阵作用在此Bethe向量的特征值。相比在Mukhin-Tarasov-Varchenko的非超对称的结果,主要的不同在于此级数是无限的,因此更为复杂。申请人另辟蹊径,巧妙地将问题转化为类似于非超对称的情形,从而证明了这一猜想。申请人进一步通过取超杨代数的相伴分次代数,证明了在超Gaudin模型(XXX自旋链的退化情形)中类似的猜想。这一结果发表在《J. High Energy Phys.》。


%$\bm Q(u)=(Q_i(u))$,这些多项式的根正好是某些Bethe变量。这里分配Bethe变量到多项式的规则取决于对应李代数的根系。申请人和导师Mukhin教授的关于超XXX自旋链一个重要的结果是,在非退化情形,求解Bethe方程等价于求解一系列关于$\bm Q(u)$的Wronskian方程。这些Wronskian方程通常被称为QQ系统,其中的Q对应Baxter的Q算子,满足著名的Baxter的TQ关系。当Bethe方程满足时,多项式$Q_i$代表$Q$算子在对应的Bethe向量的特征值。



\textbf{B. Bethe代数与其几何应用}

量子可积系统的Bethe 代数是一个由一系列作用在该系统的相空间的可交换的线性算子生成的可交换代数。比方说在XXX自旋链情形,Bethe代数是由所有传递矩阵的系数生成的杨代数的交换子代数。这是一个有可数个代数独立的生成元的交换代数。而Gaudin模型,作为XXX自旋链的退化情形,其Bethe代数是$\mathrm{U}(\mathfrak{g}[z])$的一个可交换子代数。通常我们考虑Bethe代数在相空间的上的线性算子代数里的像。在前面,我们讨论了在可积系统里,一个核心问题(Bethe代数的谱问题)是找到Bethe代数的公共特征值和公共特征向量。因此,如果我们能准确地描述Bethe代数对解决这一问题非常重要。特别地,一个非常有意思的问题是将Bethe代数等同于一个合适的概型上的函数代数。这样的等同可以认为是几何Langlands对应的一个例子。
%利用Bethe ansatz解决Bethe 代数的谱问题时,一个很关键的问题叫做Bethe ansatz的完备性:通过Bethe ansatz能不能找到所有Bethe 代数在这个量子群表示$V$里的一组(公共)特征基。通过前面的讨论,我们知道退化情形的Bethe解是无法用传统的Bethe ansatz来求解,而通过QQ系统是多项式方程从而可以解决这一问题。

Bethe代数的谱问题和几何中的Schubert 分析息息相关。事实上在研究A型Gaudin模型中,Mukhin-Tarsov-Varchenko发现当Bethe代数作用在$\mathfrak{gl}_N[z]$的赋值模的张量积的重数空间(multiplicity space)时,它对应的概型同构于一些Grassmannian里Schubert簇(取决于对应的赋值模)的交集,并且Bethe代数的作用同构于其余正则表示。从而解决了Bethe代数的谱问题,并得出Bethe代数在赋值参数全为实数时可同时对角化并具有单谱(特征值无重数)。利于它们与Schubert簇的关系,Mukhin-Tarasov-Varchenko证明了Shapiro-Shapiro猜想和对应Schubert簇交集的横截性[Ann.\,Math.\,09, JAMS 09]。进一步利用Gaudin模型和Grassmannian的关系,Mukhin-Tarasov给出了Grassmannian里Schubert分析里实解的个数的下界[Acta Math. 14]。\textbf{基于Gaudin模型在Grassmannian的Schubert分析的重要性,很自然的问题是研究其他型的Gaudin模型的谱问题及其在代数几何的应用}。Feigin-Frenkel-Rybnikov和Rybnikov证明了当Bethe代数作用在重数空间上,存在一个循环向量,从而保证当Bethe代数可对角化时其谱必须是单的。申请人在他们的工作上进一步证明了,Bethe代数作用重数空间上,是一个Frobenius代数,并且该作用同时同构于Bethe代数的像的正则表示和余正则表示。这些性质被申请人总结为可积系统的\textbf{完美可积性}。作为推论,我们得到Bethe代数的每个公共特征值只有一个约当块,即\textbf{Bethe代数的公共特征空间总是一维的}。注意在多个可交换算子作用下,循环作用并不能保证公共特征空间总是一维的。另外,在ABCG型Gaudin模型,申请人与Mukin和Varchenko教授发现其对应Bethe向量与Grassmannian里的特殊点有一一对应。我们进一步引进了Grassmannian的基于A型李代数的表示论的一个新分层。在BC型时,我们引进了Grassmannian的一个新子簇,对偶Grassmannian,并给出其相应BC型李代数表示论的分层。这些工作发表在了《SIGMA》和《 Pure Appl. Math. Q》(该卷为纪念国际著名数学家Manin教授80岁生日)上。

相比Gaudin模型,XXX自旋链要复杂的多因而相对结果比较少,因为它们的对称是由量子群来刻画。更进一步,超对称的XXX自旋链相关的结果几乎为空白,特别是奇根的出现导致一些在对称情形不存在新特征与现象。申请人和Mukhin教授对有超杨代数 $\mathrm{Y}(\mathfrak{gl}_{1|1})$对称的XXX自旋链做了非常系统地研究。这种情形有其特殊的重要意义,首先它足够简单但同时又因为它\textbf{包含了超对称的新特征与现象而足够复杂,可以给理论和猜想提供一个非常好的测试环境}。我们的主要结果是\textbf{证明了该自旋链的完美可积性}。更具体地,我们用生成元和关系式描述了Bethe代数作用在重数空间的像,从而证明了该像为Frobenius代数。进一步,我们证明了Bethe代数在上面的作用是循环地。我们描述了Bethe代数作用的公共特征值,构造对应的特征向量,以及计算对应的约当块的大小。总之,我们给出了超杨代数 $\mathrm{Y}(\mathfrak{gl}_{1|1})$对称的XXX自旋链的\textbf{Bethe代数谱问题一个完善的结果}。这一工作发表在《Comm. Math. Phys.》。
%我们发现在此情形,Bethe方程对应的QQ系统等价于两个多项式之间的整除性,从而Bethe解组成其中一个多项式$\varphi(u)$的解集的子集。这里$\varphi(u)$是由对应的表示$V$决定。在一般非退化情形,即$\varphi(u)$无重根,Bethe ansatz可以完美解决此情形的超XXX自旋链。但是如果$\varphi(u)$有重根时,我们 这一工作发表在《Comm. Math. Phys.》。

{\color{MsBlue} \subsection{\sihao \kaishu \qquad \ 
\textbf{(二)全职回国(来华)后拟开展的研究工作} {\normalfont(建议不超过4000字)} 
\bfseries \xiaosihao \fontspec{Times New Roman} 
The research work to 
be carried out after returning/coming to China full-time(no more than 4000 words)} 
}

 

{\sihao \color{MsBlue} \kaishu 主要阐述全职回国(来华)后主要研究方向和思路、
预期目标、团队和科研条件的支撑情况。}

\medskip

{\color{MsBlue} \fontspec{Times New Roman} 
In this part, you shall mainly expound the main research direction and ideas, 
expected goals, team and research conditions after returning/coming to China 
full-time.}

\medskip

申请人回国后将在目前工作基础上继续\textbf{表示论和量子可积系统}领域中的一系列核心问题做深入的研究,包括的扭杨代数的Drinfeld实现、抛物实现、极小实现及其与有限W代数和仿射Grassmannian切片的联系,与有边界的可积系统的应用。申请人未来主要研究方向简介如下:

\textbf{\sihao 1、Shifted扭杨代数和有限W代数}

有限W代数是李理论里面非常重要的一类结合代数。他取决于$(\mathfrak g,e)$,其中$\mathfrak g$是一个有限单李代数而$e$是它的一个幂零元。有限W代数被广泛的应用于研究李代数的本原理想的分类和模表示。另外,有限W代数是Slodowy切片的量子化,因此也和辛几何密切相关。

尽管有限W代数具有诸多应用,但关于其显式实现(生成元和关系式)的研究进展较为缓慢。第一个重大进展是Ragoucy和Sorba的结果,他们证明了在A型矩阵幂零元(约当块的大小一致时),有限W代数同构与A型杨代数的的截断。这份工作被Brundan和Kleshchev推广到了A型任意幂零元的情形。他们通过引进shifted杨代数,一类杨代数的子代数,来证明A型有限W代数同构于shifted 杨代数的截断。他们实现这一结果的主要工具是杨代数的抛物实现和baby comultiplication。

一个非常自然的问题是,对于其他典型的有限W代数,有没有类似的结果?实际上,在泊松代数层次,Ragoucy证明了在BCD型对应矩阵幂零元的Slodowy切片可以通过扭杨代数的截断来显示实现。这个结果最近被Tappeiner和Topley利用我们的扭杨代数的Drinfeld实现推广到许多更一般的幂零元情形。但是,推广Brundan和Kleshchev的结果到BCD型有限W代数并不是简单的平行的推广。在此,只有Brundan的学生Brown在09年关于矩阵幂零元情形及24年关于两个约当块的偶幂零元时的推广,以及De Sole-Kac-Valeri的部分结果。\textbf{申请人打算与Topley教授和他的学生Tappeiner,彭勇宁教授及王伟强教授利用扭杨代数给出对更一般的幂零元的典型有限W代数的具体实现}。

\textbf{主要思路}:首先我们将通过申请人与合作者近期关于扭杨代数的高斯分解及Drinfeld实现来引入扭杨代数的抛物实现和shifted扭杨代数,从而进一步定义baby comultiplication和shifted扭杨代数的截断。不同于Brundan-Kleshchev的构造法,我们将\textbf{采用全新的几何方法},即利用Losev的一个重要结果---Slodowy切片的过滤量子化的唯一性---来完成这一结果。这样我们将问题极大地简化到了他们的半经典极限层面。

\textbf{预期目标}: 给出扭杨代数的任意划分的抛物实现及合理定义Shifted扭杨代数的截断,并证明这些截断与BCD型关于任意偶幂零元的有限W代数时同构的,从而给出有限W代数的具体的生成元和关系式的实现方式。

\medskip

\textbf{\sihao 2、Shifted扭杨代数和仿射Grassmannian切片}

仿射Grassmannian在几何表示论里是非常重要的一个几何结构。比如在几何Satake对应中,仿射Grassmannian切片对应其Langlands对偶的最高权不可约表示的全空间。它们在数学物理中也非常重要。比如Bezrukavnikov-Finkelberg-Mirkovic 计算了仿射Grassmannian的等变Borel-Moore同调的代数结构,从而为Braverman-Finkelberg-Nakajima(BFN)后续严格数学定义Coulomb分支提供了雏形。Coulomb分支是数学物理的一个概念,而BFN的定义给辛消解提供了一系列的重要例子。人们猜测Coulomb分支与同理论的Higgs分支组成三维镜像对称对。举例来说,如果考虑箭图的规范理论,Higgs分支对应Nakajima簇而Coulomb分支对应仿射Grassmannian切片。在BFN的工作中,他们和Kamnitzer-Kodera-Webster-Weekes证明了截断shifted杨代数量子化Coulomb分支。

A型Shifted 杨代数(其shift对应支配权)最先是由Brundan-Kleshchev通过A型杨代数的抛物实现引进的。这类shifted 杨代数在后面通过杨代数的Drinfeld实现被BFN及合作者推广到了所有有限型,其中的shift对应的权可以是任意的。进一步,shifted杨代数拥有一类非常特殊的表示叫做Gerasimov-Kharchev-Lebedev-Oblezin(GKLO)表示,而其在表示里的像被定义为截断shifted杨代数。在上述讨论中,shifted杨代数和其截断给仿射Grassmannian和其切片提供了量子化,因此是一类重要的代数。另外,它们的表示论与丛代数也有密切关系。

作为杨代数的$\imath$推广,扭杨代数也拥有Drinfeld实现从而可以定义shifted 扭杨代数。\textbf{申请人打算和合作者考虑shifted twisted Yangians和仿射Grassmannian切片的关系,从而进一步探索其与Coulomb分支的联系}。

\textbf{主要思路}:扭杨代数对应对称对,是和其李代数上的对合是密切相关的。我们猜测通过在仿射Grassmannian切片上选取合适的对合再考虑此对合的固定轨迹,该固定轨迹应拥有一个泊松代数结构。我们将引入分裂及拟分裂ADE型shifted扭杨代数并构造shifted扭杨代数的GKLO型表示,从而定义截断shifted扭杨代数。另一个非常有用的工具是泊松代数的Dirac约化。我们将通过对杨代数做Dirac约化来取得与扭杨代数之间的联系。

\textbf{预期目标}:我们将证明如上引进的shifted扭杨代数将量子化一大类仿射Grassmannian切片在多种对合下的固定轨迹。在完成上面一目标后,我们将研究这些固定轨迹在何时可以通过用Coulomb分支来实现。

\medskip

\textbf{\sihao 3、扭杨代数的极小实现及不同实现之间的同构}

仿射量子群和扭杨代数的极小实现是一个非常有用的工具,这里的极小实现是Drinfeld实现的简化版本。不同于Drinfeld实现考虑所有次的生成元,极小实现只考虑Drinfeld实现中的零次和一次生成元及其对应关系。通过这些生成元来构造其他高次生成元并验证这些高次元满足对应的关系式。因为其含有更少的生成元与关系式,所以极小实现能方便地用来验证关系式,从而判断一个关于仿射量子群的映射是不是代数同态。这个发现,被景乃桓教授、张红莲教授与合作者用来证明扭仿射量子群、AB型量子仿射超代数的Drinfeld和Drinfeld-Jimbo实现之间的具体同构。另外一个重要的应用是Guay-Nakajima-Wendlandt将杨代数的极小实现推广到了Kac-Moody型的杨代数,并基于此构造了Kac-Moody型的杨代数的余积。

在前面与合作者的工作中,申请人通过对仿射$\imath$量子群的做退化给出来所有分裂型与拟分裂ADE型扭杨代数的表示。一个开放性的问题就是这样构造的扭杨代数是不是对应杨代数的余理想子代数,如果是,那么能不能描述它们在余积下的作用。这两个问题是息息相关的,前者需要考虑如何将扭杨代数嵌入到杨代数里并且计算后者来验证得到;后者的一个简洁扼要的答案对研究扭杨代数的表示至关重要。\textbf{申请人打算与合作者通过构造扭杨代数的极小实现来解决这两个问题,并利用扭杨代数的极小实现来证明扭杨代数的Drinfeld实现与R-matrix实现、Drinfeld的$\mathcal J$-实现之间的同构}。

\textbf{主要思路}:构造R-matrix实现与Drinfeld的$\mathcal J$-实现之间的具体同构相对简单,因为很容易构造一个从Drinfeld的$\mathcal J$-实现到R-matrix实现的代数同态。验证其为满射只需要计算对应的像能生成整个代数。至于验证其为单射,考虑到它们都是杨代数的子代数,我们可以利用杨代数的PBW定理来完成。而Drinfeld实现到另两者的具体同构将通过极小实现来完成。这个极小实现可以通过考虑对应的相伴分次代数来猜测并证明。

\textbf{预期目标}:构造扭杨代数的极小实现,以此构造扭杨代数三种不同实现方式的具体同构。利用具体的同构和$\mathcal J$-实现的余积公式,计算扭杨代数的余积并证明依Drinfeld实现定义的扭杨代数为对应杨代数的余理想子代数,为研究扭杨代数的表示之间的张量积提供理论基础。

\medskip



\textbf{\sihao 4、XXX自旋链的Bethe代数谱问题}

相比Gaudin模型,关于XXX自旋链的Bethe代数谱问题的非常少。其主要原因在于,XXX自旋链的代数使用杨代数来描述的,它们的表示论要比Kac-Moody李代数的表示论复杂得多。在当对应相空间为一些赋值自然表示的不可约张量积时,Mukhin-Tarasov-Varchenko证明了Bethe代数作用在上面有一个循环向量,且Bethe代数的像是一个Frobenius代数,从而解决这种特殊情形时的Bethe代数谱问题。在后续工作中,Gorbounov-Rimányi-Tarasov-Varchenko(GRTV)证明了此Bethe代数同构于偏旗簇的余切丛的等变上同调代数。

另一方面,麻省理工的Maulik教授和菲尔茨奖得主Okounkov教授描述了A型杨代数在A型Nakajima簇的局部化的等变上同调环,其中Bethe代数的一些元素的在其上的作用对应与某些上同调类的量子乘积。因此他们猜想,其对应的量子上同调环应对应Bethe代数作用在一些赋值基本(对应Young图为一列)表示的不可约张量积时。在这些表示都是自然表示时,GRTV利用Bethe代数的具体描述从而验证了这一猜想。\textbf{申请人打算与Mukhin教授、Tarasov教授将XXX自旋链的Bethe代数作用在赋值基本表示的不可约张量积的谱问题}。

\textbf{主要思路}:我们打算(1)证明Bethe代数在其上的作用是循环的并且Bethe代数是Frobenius代数,(2)从而进一步给出Bethe代数用生成元和关系式的实现。我们的主要工具分为两部分,第一部分是应用Yangians的表示论,特别是应用R-matrix的fusion积;第二部分我们将使用QQ系统,考虑其对应的差分算子的解空间的特性,这个解空间是由多项式组成,并且在赋值参数处有特殊的性质。更具体地,此时的Bethe代数应对应某个离散Wronski映射的纤维上的函数代数。

\textbf{预期目标}:给出Bethe代数作用在赋值基本表示的不可约张量积时的具体刻画,从而解决其谱问题,并为验证Maulik-Okounkov的猜想提供理论依据。进一步通过与Grassmannian的联系,来解决一系列关于特殊的旗簇的Schubert分析问题。

\medskip

\textbf{\sihao 5、有边界的Gaudin模型}

在前面的讨论中,Gaudin模型与Grassmannian有着紧密的联系,从而被用来解决一系列与Grassmannian里的重要问题。而这里的Gaudin模型是对应无边界(或者是周期边界)的量子可积系统。在著名数学物理学家Sklyanin的工作中,他揭示了如何利用反射方程来构造一系列的带边界条件的可积系统。在起初的数学物理文献里,Gaudin模型一般是通过取对应的XXX自旋链的退化来得到的,从而继续被研究。但是这一方法需要对对应的李代数的型做分类讨论,这是因为不同型的量子群其对应的最基本的R-matrix(Yang-Baxter方程的解)通常差距很大,因此很难用一个统一的方式来刻画。对应地,在有边界的量子可积系统里,这样的问题更为严重,因为它们是用Satake图或者对称对来分类,比Dynkin图的种类要多的多。

Gaudin模型是由Gaudin统一规划地通过李代数的Casimir元来引入的。通过Casimir元,Gaudin定义了二阶Gaudin哈密顿算子并证明了这些算子可交换从而构成一个可积系统。MacKay教授通过Casimir元与其对应李代数上的对合的来定义了对所有对称对的扭杨代数。这种方式的优势在于,其中的李代数可以使任意的,而其对合也是,因此具体定义的扭杨代数进取决于对称对的型。\textbf{申请人打算和Mukhin教授打算通过类似的方式来引入具有边界的Gaudin模型并研究其上的Bethe ansatz方法}。

\textbf{主要思路}:我们通过适当将Casimir元分解到对合的两个向量空间,从而引进一系列可交换的二阶Gaudin哈密顿算子,定义带边界的Gaudin模型。进一步将采取Schechtman-Varchenko在[Invent. Math. 91]和Feigin-Frenkel-Reshetikhin在[Comm. Math. Phys. 94]类似的方法来得到权向量函数并证明当参数是Bethe方程的解时,则此权向量函数正好是二阶(或高阶)Gaudin哈密顿算子的特征向量。在后续工作中,我们还打算考虑其Bethe代数的谱问题和在几何中的应用。

\textbf{预期目标}:引入对任意对称对的Gaudin模型,计算其对应的Bethe方程和对应的Bethe向量。研究对应的reproduction procedure和QQ系统,建立相应的几何Langlands对应,最终解决其Bethe代数的谱问题。

%\textbf{主要思路}:



%\medskip

%\textbf{\sihao 预期目标}

%学术研究方面,申请人希望回国后能完成上面几个拟研究课题,特别是关于有限W代数和仿射Grassmannian切片的部分。论文发表方面,申请人希望能在国际一流期刊上发表多篇论文。团队组建方面,申请人希望能培养1--2名博士后和若干名博士生,声请人希望能多组织一些学生讨论班以及开设相关的表示论与数学物理课程,培养一些学生对表示论及其应用的兴趣。同时,我希望能与国内的同事多交流讨论合作,尽可能的开展新的合作研究方向。学术交流方面,申请人也会与同事们积极组织学术研讨班和国内/国际学术会议,邀请专家们来讲述他们最新的研究进展,开拓我们的视野。申请人也会与国际专家Varchenko教授,王伟强教授,Mukhin教授等人继续开展学术交流活动。

\medskip

\textbf{\sihao 团队和科研 条件的支撑情况}

南方科技大学及深圳国际数学中心有非常不错的表示论和数学物理研究团队,有国际知名的专家和年轻学者。其中,Zelmanov教授是菲尔茨奖得主,国际数学家大会一小时报告人,国际知名表示论专家;Futorny教授是巴西科学院院士,国际数学家大会45分钟报告人,国际知名表示论专家;向子卿副教授是代数组合与表示论的专家;马梓铭副教授 和邬龙挺助理教授是代数几何和数学物理的专家;冯致程助理教授是李群及其表示论专家。他们的研究内容都与我的研究课题紧密相连或者间接相关。我相信我的加入能更深化团队合作,壮大研究队伍。南方科技大学将提供给我非常不错的研究条件和支持。

\bigskip

{\color{MsBlue} \subsection{\sihao \kaishu \qquad \ \bfseries(三)其他需要说明的情况 
\xiaosihao \fontspec{Times New Roman} Other issues need to be addressed.}}
%

\bigskip

{\sihao \color{MsBlue} \kaishu 1.申请人同年申请不同类型的国家自然科学基金项目情况(列明同年申请的其他项目的项目类型、项目名称信息,并说明与本项目之间的区别与联系;已收到自然科学基金委不予受理或不予资助决定的,无需列出)。}

\bigskip

{\color{MsBlue} \xiaosihao \fontspec{Times New Roman} 
Proposals that the applicant has submitted for different types of NSFC programs in the same year (please list the types of programs and title of proposals submitted in the same year, and explain the differences and connections with this proposal;Proposals deemed ineligible or unfundable by the NSFC can be excluded).}

\bigskip

无。

\bigskip

{\sihao \color{MsBlue} \kaishu 2.申请人是否存在同年申请或者参与申请国家自然科学基金项目的单位不一致的情况(如存在上述情况,列明所涉及人员的姓名,申请或参与申请的其他项目的项目类型、项目名称、单位名称、上述人员在该项目中是申请人还是参与者,并说明单位不一致原因)。}

\bigskip

{\color{MsBlue} \xiaosihao \fontspec{Times New Roman} 
Whether the applicant's host institution is inconsistent with the one indicated in other proposals that he or she submits or participates in applying in the same year (if there is any such inconsistency, please list the names of the personnel involved, the types of programs, titles of proposals, names of host institutions for other projects that you applied or participated in, whether the abovementioned personnel are the applicants or participants in the projects, and explain the reasons for the inconsistency).}

\bigskip

无。

\bigskip

{\sihao \color{MsBlue} \kaishu 3.申请人是否存在与正在承担的国家自然科学基金项目的单位不一致的情况(如存在上述情况,列明所涉及人员的姓名,正在承担项目的批准号、项目类型、项目名称、单位名称、起止年月,并说明单位不一致原因)。}

\bigskip

{\color{MsBlue} \xiaosihao \fontspec{Times New Roman} 
Whether the applicant's host institution is inconsistent with the one indicated in the NSFC project that he/she is undertaking (if there is any such inconsistency, please list the name of the personnel involved, approval number, type of program, title of proposal, name of host institution, start and end dates of the undertaking project, and explain the reasons for the inconsistency).}

\bigskip

无。

\bigskip

{\sihao \color{MsBlue} \kaishu 4.申请人教育或工作经历若不连续请说明原因。}

\bigskip

{\color{MsBlue} \xiaosihao \fontspec{Times New Roman} 
If there is any discontinuity in education or work experience, please explain the reason.}

\bigskip

申请人因本科期间荒废部分学业,于毕业后调整修养后决定考研继续深造,在2010年八月至2012年八月期间并无稳定正式工作。

\bigskip

{\sihao \color{MsBlue} \kaishu 5. 同年以不同专业技术职务(职称)申请或参与申请科学基金项目的情况(应详细说明原因)。 

\bigskip

\xiaosihao \fontspec{Times New Roman} Situation where the applicant applies for NSFC programs as PI or participant using different professional or academic titles in the same year (Please elaborate on the reasons).}

\bigskip

无。

\bigskip

{\sihao \color{MsBlue} \kaishu 6. 其他。 
\xiaosihao \fontspec{Times New Roman} Others.}

\bigskip

无。

\bigskip

\end{document}


