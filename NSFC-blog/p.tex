% 国家自然科学基金NSFC海外优青申请书正文模板(2024版)

% 声明:
% 注意!!!非国家自然科学基金委官方模版!!!由个人根据官方MsWord模版制作。本模版的
% 作者尽力使本模版和官方模版生成的PDF文件视觉效果大致一样,然而,并不保证本模版有用,
% 也不对使用本模版造成的任何直接或间接后果负责。 强烈建议自己对照官方MsWord模板确认格式
% 和文字是否一致,尤其是蓝字部分。

%本模版可以自由修改以满足用户自己的需要。但是如果要传播本模版,则只能传播未经修改的版本。
%不得将本模版用于商用或获取经济利益。
%使用本模版意味着同意上述声明。


%默认小四号字。允许楷体粗体。
\documentclass[12pt,UTF8,AutoFakeBold=4,a4paper]{ctexart} 
\usepackage{amsmath,amssymb,graphicx,mathrsfs}
\include{pkgs}
\input{cmds}


%%%% 正文开始 %%%%
\begin{document}

\begin{center}
{\sanhao \CJKfontspec{SimHei} 报告正文 
\bfseries \fontspec{Times New Roman} Main Body of Proposal}
\end{center}

{\sihao \kaishu  
参照以下提纲撰写,要求内容翔实、清晰,层次分明,标题突出。}

{\sihao \fontspec{Times New Roman} The proposal shall be written in accordance 
with the following outline, with informative content, clear structure, 
and prominent titles}

{\sihao \kaishu \color{MsBlue} \bfseries 请勿删除或改动下述提纲标题及括号中的文字。}

{\sihao \color{MsBlue} \fontspec{Times New Roman} Please do not delete or 
change the title of the outline and the words in brackets.}

% \vskip -5mm


{\color{MsBlue} \subsection{\sihao \kaishu \quad \ 
\textbf{(一)主要学术成绩}(建议不超过4000字)}}

{\color{MsBlue} \xiaosihao \fontspec{Times New Roman} 
\textbf{Major academic achievements (no more than 4000 words)}}

{\sihao \kaishu \color{MsBlue} 着重阐述所取得研究成果的创新性、科学价值及本人贡献等。}

{\color{MsBlue} \xiaosihao \fontspec{Times New Roman} In this part, you shall 
focus on the innovativeness and scientific value of the research results, 
and your personal contribution.}

% \newpage
% \setlength{\bibsep}{0.0pt}
% \bibliographystyle{gbt7714-numerical}
% \bibliography{p}
% \newpage

%可以通过类似的命令微调行距以使得排版美观
% \vskip -5mm 

本人在Comm. Math. Phys.,JHEP等国际一流期刊发表和接收了17篇文章。被ICM45分钟报告人、Whitehead奖得主、英国约克大学教授Nazarov和澳大利亚科学院院士、悉尼大学教授Molev等国际顶尖数学家多次引用。担任Compos. Math.,Comm. Math. Phys.,Sel. Math. New Ser.,J. Ec. Polytech. - Math.,IMRN等多个国际期刊审稿人。

\medskip

\textbf{\sihao 1、Twisted Yangian的Drinfeld实现}

Yangians最早出现在数学物理中,是在 80 年代 Faddeev 及其学派关于量子反散射方法的研究中。更一般的Yangians是由Drinfeld在1985年引进用来给Yang-Baxter方程提供有理函数解的、非常重要的一类量子群。它们在数学物理和表示论有着非常广泛的应用。Twisted Yangians是Yangians的一类重要的余理想子代数。不同于Yangians是用Dynkin图来分类的,twisted Yangians是用对称对或者Satake图来分类的。它们(AI和AII型)最先是由Olshanski在1990年通过R-matrix和Cherednik的reflection方程来引进的。另外,twisted Yangians有到BCD型李代数的赋值同态,因此twisted Yangians比BCD型Yangians与BCD型李代数的表示论的联系更加紧密。

Yangians和仿射量子群的Drinfeld(新)实现是由Drinfeld在1988年引进的。Drinfeld实现对于引进$q$-特征,研究仿射量子群的表示论,以及引进shifted Yangians至关重要。因为李代数可以看作是对角型的对称对,因此可以认为对称对是李代数的推广,进一步twisted Yangians是Yangians的推广。因此,一个非常自然且重要的问题就是twisted Yangians是不是也有Drinfeld实现。\textbf{这是一个长期悬而未决的公开问题},甚至在卢明教授和王伟强教授给出仿射$\imath$量子群的Drinfeld实现之前,人们可能认为这样的Drinfeld实现并不存在。

申请人与弗吉尼亚大学的王伟强教授,香港大学的张伟南博士(arXiv:2308.12254)通过R-matrix实现做高斯分解的方法\textbf{首次得到了分裂型A型(即AI型)twisted Yangians的Drinfeld实现}。这份工作开启了\textbf{发现twisted Yangians的Drinfeld实现的大门}。在后续的工作中(arXiv:2406.05067,arXiv:2408.06981),申请者与王伟强教授,张伟南博士对仿射$\imath$量子群的Drinfeld实现做退化,从而\textbf{得到了所有分裂型和拟分裂ADE型}twisted Yangians的Drinfeld实现。我们的创新点在于这两种方法独立却又互补。一方面高斯分解帮助我们确定如何在仿射$\imath$量子群那边选取合适的参数做退化。另一方面通过退化仿射$\imath$量子群的Serre关系式,我们得到twisted Yangians的Serre关系式,而这个Serre关系式单独从高斯分解是很难得出一个简单而又优美的公式。

Twisted Yangians的Drinfeld实现将提供一系列的应用,比如研究Twisted Yangians表示论的$q$-特征理论,引进shifted twisted Yangians并研究其与几何中Slodowy切片、有限W代数、仿射Grassmannian切片的联系。比如Topley教授和其学生Tappeiner 最近利用我们的 twisted Yangians 的 Drinfeld实现研究典型李代数的Slodowy切片,从而 极大地推广了Topley教授在2023年发表在《Invent. Math.》的结果。

我们其中一份工作(arXiv:2406.05067)已被《Comm. Math. Phys.》接受,得到了两位审稿人的高度评价:``\textbf{addresses an important problem} in the representation theory of affine
quantum symmetric pairs that will help \textbf{bring the theories of twisted Yangians and affine ıquantum groups to a level playing field}",``\textbf{was an important open problem for many years} which has been \textbf{resolved in the present paper in many cases}"。

\medskip

\textbf{\sihao 2、仿射$\imath$量子群的R-matrix和Drinfeld实现}

仿射量子群通常有三种常用的实现方式,即Drinfeld-Jimbo、Drinfeld、和R-matrix实现。不同的实现方式有不同的优势,有些实现方式在研究特定问题如可积系统的对称性、表示论、或与几何之间的联系比其他实现方式更方便。比如说Drinfeld实现是研究仿射量子群的表示论主要工具,它能用来定义$q$-特征,统一地给有限维不可约表示进行分类,定义截断shifted仿射量子群(包括Yangians)并研究其与仿射Grassmannian切片(或更进一步与K-理论版本的Coulomb分支)的联系等。另一方面,Drinfeld-Jimbo和R-matrix实现相比Drinfeld实现能更好地描述仿射量子群的Hopf代数结构,特别是它们的余积(coproduct)。因此\textbf{找到仿射量子群不同实现方式的之间的直接同构与联系是非常重要且有用的工作}。这一系列工作由Beck,Ding-Frenkel,Damiani,Jing-Liu-Molev等教授完成。

Twisted $q$-Yangians 是Molev-Ragoucy-Sorba通过类似Olshanski的方法用R-matrix实现构造的A型仿射量子群的余理想子代数。他们对应的对称对(或Satake图)是AI和AII型。更一般的仿射$\imath$量子群,作为仿射量子群的余理想子代数,是由Kolb教授[Adv. Math. 2014]通过Drinfeld-Jimbo实现推广Letzter教授对于有限型量子对称对而引进的。并且Kolb教授证明了AI和AII型仿射$\imath$量子群分别同构于对应的twisted $q$-Yangians并给出Drinfeld-Jimbo实现和R-matrix实现的生成元之间的对应。最近,卢明教授和王伟强教授[Adv. Math. 2021]通过仿射$\imath$量子群的辫子群作用构造了一系列新的生成元从而得到了分裂ADE型仿射$\imath$量子群的Drinfeld实现,推广前面Beck和Damiani的工作。这一工作被张伟南博士推广到了分裂BCFG型。另外,卢明教授,王伟强教授和张伟南博士进一步推广到了拟分裂ADE型。他们的这些工作完成了仿射$\imath$量子群的Drinfeld-Jimbo实现和Drinfeld实现之间详细的同构与联系。但是,仿射$\imath$量子群的R-matrix实现和Drinfeld实现之间的同构仍是缺失的。

申请人通过结合Ding-Frenkel早期关于A型仿射量子群的工作和最近与王伟强教授和张伟南博士关于twisted Yangians的工作,\textbf{首次给出了仿射$\imath$量子群(AI型)的R-matrix实现和Drinfeld实现的同构并给出了生成元之间的对应},填补了这一空白。这一工作将为未来构造其他类型仿射$\imath$量子群的R-matrix和Drinfeld实现的同构提供指导意见。该结果发表在《Int.
Math. Res. Not.》,并得到了审稿人的好评“\textbf{presents crucial and important results}”“I \textbf{strongly recommend} it for publication in IMRN”。

\medskip

\textbf{\sihao 3、超Yangians的表示论及Jacobi-Trudi等式}

超对称是现代理论物理非常重要的一个理论框架,而超代数是描述超对称的数学框架。这里超代数是一个$\mathbb Z_2$-分次代数,可以分解为奇、偶两部分。超Yangians是由ICM45分钟报告人Nazarov教授引入的,关于Yangians的自然推广,它们在表示论和里数学物理非常重要,有着非常广泛的应用。例如,国立中山大学的彭勇宁教授用超Yangians及其子代数Shifted超Yangians来刻画有限W超代数的实现;Nazarov教授利用它的量子超行列式来得到线性李超代数的Capelli等式;



该结果发表在《Int. Math. Res. Not.》,并得到了审稿人的好评“The main result (Theorem 5.12) is certainly \textbf{interesting and
beautiful}” “it is also \textbf{novel and not an obvious generalization} of a known
result for $\mathrm{Y}(\mathfrak{gl}_m)$”。


\medskip

\textbf{\sihao 4、Bethe方程的解和QQ关系式}

在研究量子可积系统中,一个非常重要的问题是找到传递矩阵的

\medskip

\textbf{\sihao 5、高阶传递矩阵的特征值}

这一工作发表在《J. High Energy Phys.》。

{\color{MsBlue} \subsection{\sihao \kaishu \quad \ 
\textbf{(二)全职回国(来华)后拟开展的研究工作} (建议不超过4000字) 
\bfseries \xiaosihao \fontspec{Times New Roman} 
The research work to 
be carried out after returning/coming to China full-time(no more than 4000 words)} 
}

{\sihao \color{MsBlue} \kaishu 主要阐述全职回国(来华)后主要研究方向和思路、
预期目标、团队和科研条件的支撑情况。}

{\color{MsBlue} \fontspec{Times New Roman} 
In this part, you shall mainly expound the main research direction and ideas, 
expected goals, team and research conditions after returning/coming to China 
full-time.}

\medskip

\textbf{\sihao 1、Shifted twisted Yangian和有限W代数}

有限W代数是李理论里面一类非常重要重要的结合代数。他取决于$(\mathfrak g,e)$,其中$\mathfrak g$是一个有限单李代数而$e$是它的一个幂零元。有限W代数被广泛的应用于研究李代数的本原理想的分类和模表示。另外,有限W代数是Slodowy切片的量子化,因此他们也和辛几何密切相关。

尽管有限W代数具有诸多应用,但关于其显式实现(生成元和关系式)的研究进展较为缓慢。第一个重大进展是Ragoucy和Sorba的结果,他们证明了在A型并且幂零元有$N$个大小均为$\ell$的约当块时,有限W代数同构与A型Yangian $\mathrm{Y}(\mathfrak{gl}_N)$的层级为$\ell$的截断。这份工作被Brundan和Kleshchev推广到了A型任意幂零元的情形。他们通过引进shifted Yangians,一类Yangian的子代数,来证明A型有限W代数同构于shifted Yangians的截断。他们实现这一结果的主要工具是Yangian的抛物实现和baby comultiplication。

一个非常自然的问题是,对于其他典型的有限W代数,有没有类似的结果?实际上,在泊松代数层次,Ragoucy证明了在BCD型Slodowy切片可以通过twisted Yangian的截断来显示实现。这个结果最近被Tappeiner和Topley利用我们的twisted Yangians的Drinfeld实现推广到许多更一般的幂零元情形。但是,在量子层面,也就是有限W代数,只有部分结果被证明了,其中包括De Sole–Kac–Valeri和Brown的结果。

申请人打算与Topley教授和他的学生Tappeiner,彭勇宁教授及王伟强教授推广Brundan和Kleshchev的结果到对应\textbf{任意偶幂零元}的BCD型有限W代数。也就是说,当$e$是任意一个偶幂零元(这里的偶对应定义W代数的分次)且$\mathfrak g$是BCD型李代数时,有限W代数同构于shifted twisted Yangians的截断。这样,有限W代数就拥有了具体的生成元和关系式的实现方式。

\textbf{主要思路}:推广Brundan和Kleshchev的结果到BCD型并不是简单的平行的推广。实际上Brundan的学生Brown花了很大一部分时间来实现这个推广,但是很不幸的是,他的结果只能应用到幂零元的约当块都是同样大小的情形。即使是两个约当块的偶幂零元,这个结果也是Brown正在进行的工作。我们将采用新的几何方法,利用Losev的重要结果---Slodowy切片的过滤量子化的唯一性---来完成这一结果。这样我们将问题极大地简化到了他们的半经典极限层面。当然我们还是需要另外的两个工具:抛物实现和baby comultiplication。

\medskip

\textbf{\sihao 2、Shifted twisted Yangian和仿射Grassmannian切片}

\medskip

\textbf{\sihao 3、有边界的可积系统}

\textbf{主要思路}:


{\color{MsBlue} \subsection{\sihao \kaishu \quad \ \bfseries(三)其他需要说明的情况 
\xiaosihao \fontspec{Times New Roman} Other issues need to be addressed.}}
%

{\sihao \color{MsBlue} \kaishu 1.申请人同年申请不同类型的国家自然科学基金项目情况(
列明同年申请的其他项目的项目类型、项目名称信息,并说明与本项目之间的区别与联系)。}

{\color{MsBlue} \xiaosihao \fontspec{Times New Roman} 
The applications of the applicant for different types of NSFC programs in the 
ame year (please list the types of programs and title of proposals submitted 
in the same year, and explain the difference and connection with this proposal).}

无。

{\sihao \color{MsBlue} \kaishu 2.申请人是否存在同年申请或者参与申请国家自然科
学基金项目的单位不一致的情况(如存在上述情况,列明所涉及人员的姓名,申请或参与申请的其他
项目的项目类型、项目名称、单位名称、上述人员在该项目中是申请人还是参与者,并说明单位不一
致原因)。}

{\color{MsBlue} \xiaosihao \fontspec{Times New Roman} 
Whether the applicant's host institution is inconsistent with the one indicated 
in other proposals that he or she submits or participates in applying in the 
same year (if there is any such inconsistency, please list the names of the 
personnel involved, the types of programs, titles of proposals, names of host 
institutions for other projects that you applied or participated in, whether 
the abovementioned personnel are the applicants or participants in the projects, 
and explain the reasons for the inconsistency).}

无。

{\sihao \color{MsBlue} \kaishu 3.申请人是否存在与正在承担的国家自然科学基金项目的单位
不一致的情况(如存在上述情况,列明所涉及人员的姓名,正在承担项目的批准号、项目类型、
项目名称、单位名称、起止年月,并说明单位不一致原因)。}

{\color{MsBlue} \xiaosihao \fontspec{Times New Roman} 
Whether the applicant's host institution is inconsistent with the one indicated 
in the NSFC project that he/she is undertaking (if there is any such inconsistency, please list the name of the personnel involved, approval number, type of pro
gram, title of proposal, name of host institution, start and end dates of the un
dertaking project, and explain the reasons for the inconsistency).}

无。

{\sihao \color{MsBlue} \kaishu 4.申请人教育或工作经历若不连续请说明原因。}

{\color{MsBlue} \xiaosihao \fontspec{Times New Roman} 
If there is any discontinuity in education or work experience, please explain 
the reason.}

{\sihao \color{MsBlue} \kaishu 5. 其他。 
\xiaosihao \fontspec{Times New Roman} Others.}


无。


\end{document}


