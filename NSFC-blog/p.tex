% 国家自然科学基金NSFC海外优青申请书正文模板(2024版)

% 声明:
% 注意!!!非国家自然科学基金委官方模版!!!由个人根据官方MsWord模版制作。本模版的
% 作者尽力使本模版和官方模版生成的PDF文件视觉效果大致一样,然而,并不保证本模版有用,
% 也不对使用本模版造成的任何直接或间接后果负责。 强烈建议自己对照官方MsWord模板确认格式
% 和文字是否一致,尤其是蓝字部分。

%本模版可以自由修改以满足用户自己的需要。但是如果要传播本模版,则只能传播未经修改的版本。
%不得将本模版用于商用或获取经济利益。
%使用本模版意味着同意上述声明。


%默认小四号字。允许楷体粗体。
\documentclass[12pt,UTF8,AutoFakeBold=2,a4paper]{ctexart} 
\usepackage{amsmath,amssymb,graphicx,mathrsfs}
\include{pkgs}
\input{cmds}


%%%% 正文开始 %%%%
\begin{document}

{\sanhao \CJKfontspec{SimHei} 报告正文 
\bfseries \fontspec{Times New Roman} Main Body of Proposal}

{\sihao \kaishu  
参照以下提纲撰写,要求内容翔实、清晰,层次分明,标题突出。}

{\sihao \fontspec{Times New Roman} The proposal shall be written in accordance 
with the following outline, with informative content, clear structure, 
and prominent titles}

{\sihao \kaishu \color{MsBlue} \bfseries 请勿删除或改动下述提纲标题及括号中的文字。}

{\sihao \color{MsBlue} \fontspec{Times New Roman} Please do not delete or 
change the title of the outline and the words in brackets.}

% \vskip -5mm


{\color{MsBlue} \subsection{\sihao \kaishu \quad \ 
\textbf{(一)主要学术成绩}(建议不超过4000字)}}

{\color{MsBlue} \xiaosihao \fontspec{Times New Roman} 
\textbf{Major academic achievements (no more than 4000 words)}}

{\sihao \kaishu \color{MsBlue} 着重阐述所取得研究成果的创新性、科学价值及本人贡献等。}

{\color{MsBlue} \xiaosihao \fontspec{Times New Roman} In this part, you shall 
focus on the innovativeness and scientific value of the research results, 
and your personal contribution.}

% \newpage
% \setlength{\bibsep}{0.0pt}
% \bibliographystyle{gbt7714-numerical}
% \bibliography{p}
% \newpage

%可以通过类似的命令微调行距以使得排版美观
% \vskip -5mm 

\textbf{\sihao 一、Twisted Yangian的Drinfeld实现}

{\color{MsBlue} \subsection{\sihao \kaishu \quad \ 
\textbf{(二)全职回国(来华)后拟开展的研究工作} (建议不超过4000字) 
\bfseries \xiaosihao \fontspec{Times New Roman} 
The research work to 
be carried out after returning/coming to China full-time(no more than 4000 words)} 
}

{\sihao \color{MsBlue} \kaishu 主要阐述全职回国(来华)后主要研究方向和思路、
预期目标、团队和科研条件的支撑情况。}

{\color{MsBlue} \fontspec{Times New Roman} 
In this part, you shall mainly expound the main research direction and ideas, 
expected goals, team and research conditions after returning/coming to China 
full-time.}

\textbf{\sihao 1、Shifted twisted Yangian和有限W-代数}

有限W代数是李理论里面一类非常重要重要的结合代数。他取决于$(\mathfrak g,e)$,其中$\mathfrak g$是一个有限单李代数而$e$是它的一个幂零元。有限W代数被广泛的应用于研究李代数的本原理想的分类和模表示。另外,有限W代数是Slodowy切片的量子化,因此他们也和辛几何密切相关。

尽管有限W代数具有诸多应用,但关于其显式实现(生成元和关系式)的研究进展较为缓慢。第一个重大进展是Ragoucy和Sorba的结果,他们证明了在A型并且幂零元有$N$个大小均为$\ell$的约当块时,有限W代数同构与A型Yangian $\mathrm{Y}(\mathfrak{gl}_N)$的层级为$\ell$的截断。这份工作被Brundan和Kleshchev推广到了A型任意幂零元的情形。他们通过引进shifted Yangians,一类Yangian的子代数,来证明A型有限W代数同构于shifted Yangians的截断。他们实现这一结果的主要工具是Yangian的抛物实现和baby 

一个非常自然的问题是,对于其他典型的有限W代数,有没有类似的结果?实际上,在泊松代数层次,Ragoucy证明了在BCD型Slodowy切片可以通过twisted Yangian的截断来显示实现。这个结果最近被Tappeiner和Topley利用我们的twisted Yangians的Drinfeld实现推广到许多更一般的幂零元情形。但是,在量子层面,也就是有限W代数,只有部分结果被证明了,其中包括De Sole–Kac–Valeri和Brown的结果。

申请人打算与Topley教授和他的学生Tappeiner,彭勇宁教授及王伟强教授推广Brundan和Kleshchev的结果到对应\textbf{任意偶幂零元}的BCD型有限W代数。也就是说,当$e$是任意一个偶幂零元(这里的偶对应定义W代数的分次)且$\mathfrak g$是BCD型李代数时,有限W代数同构于shifted twisted Yangians的截断。这样,有限W代数就拥有了具体的生成元和关系式的实现方式。

\textbf{主要思路}:推广Brundan和Kleshchev的结果到BCD型并不是简单的平行的推广。实际上Brundan的学生Brown花了很大一部分时间来实现这个推广,但是很不幸的是,他的结果只能应用到幂零元的约当块都是同样大小的情形。即使是两个约当块的偶幂零元,这个结果也是Brown正在进行的工作。我们将采用新的几何方法,利用Losev的重要结果---Slodowy切片的过滤量子化的唯一性---来完成这一结果。这样我们将问题极大地简化到了他们的半经典极限层面。当然我们还是需要另外的两个工具:抛物实现和baby comultiplication。

\textbf{\sihao 2、Shifted twisted Yangian和仿射Grassmannian切片}

\textbf{\sihao 3、有边界的可积系统}

\textbf{主要思路}:


{\color{MsBlue} \subsection{\sihao \kaishu \quad \ \bfseries(三)其他需要说明的情况 
\xiaosihao \fontspec{Times New Roman} Other issues need to be addressed.}}
%

{\sihao \color{MsBlue} \kaishu 1.申请人同年申请不同类型的国家自然科学基金项目情况(
列明同年申请的其他项目的项目类型、项目名称信息,并说明与本项目之间的区别与联系)。}

{\color{MsBlue} \xiaosihao \fontspec{Times New Roman} 
The applications of the applicant for different types of NSFC programs in the 
ame year (please list the types of programs and title of proposals submitted 
in the same year, and explain the difference and connection with this proposal).}

% 无。

{\sihao \color{MsBlue} \kaishu 2.申请人是否存在同年申请或者参与申请国家自然科
学基金项目的单位不一致的情况(如存在上述情况,列明所涉及人员的姓名,申请或参与申请的其他
项目的项目类型、项目名称、单位名称、上述人员在该项目中是申请人还是参与者,并说明单位不一
致原因)。}

{\color{MsBlue} \xiaosihao \fontspec{Times New Roman} 
Whether the applicant's host institution is inconsistent with the one indicated 
in other proposals that he or she submits or participates in applying in the 
same year (if there is any such inconsistency, please list the names of the 
personnel involved, the types of programs, titles of proposals, names of host 
institutions for other projects that you applied or participated in, whether 
the abovementioned personnel are the applicants or participants in the projects, 
and explain the reasons for the inconsistency).}

% 无。

{\sihao \color{MsBlue} \kaishu 3.申请人是否存在与正在承担的国家自然科学基金项目的单位
不一致的情况(如存在上述情况,列明所涉及人员的姓名,正在承担项目的批准号、项目类型、
项目名称、单位名称、起止年月,并说明单位不一致原因)。}

{\color{MsBlue} \xiaosihao \fontspec{Times New Roman} 
Whether the applicant's host institution is inconsistent with the one indicated 
in the NSFC project that he/she is undertaking (if there is any such inconsistency, please list the name of the personnel involved, approval number, type of pro
gram, title of proposal, name of host institution, start and end dates of the un
dertaking project, and explain the reasons for the inconsistency).}

% 无。

{\sihao \color{MsBlue} \kaishu 4.申请人教育或工作经历若不连续请说明原因。}

{\color{MsBlue} \xiaosihao \fontspec{Times New Roman} 
If there is any discontinuity in education or work experience, please explain 
the reason.}

{\sihao \color{MsBlue} \kaishu 5. 其他。 
\xiaosihao \fontspec{Times New Roman} Others.}


% 无。


\end{document}


